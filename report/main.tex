\documentclass[12pt]{article}
\usepackage[english]{babel}
\usepackage{natbib}
\usepackage{url}
\usepackage[utf8x]{inputenc}
\usepackage{amsmath}
\usepackage{graphicx}
\graphicspath{{image/}}
\usepackage{parskip}
\usepackage{fancyhdr}
\usepackage{vmargin}
\usepackage{listings}
\usepackage{color}

\setmarginsrb{3 cm}{2.5 cm}{3 cm}{2.5 cm}{1 cm}{1.5 cm}{1 cm}{1.5 cm}

\usepackage{listings}
\usepackage{color}
\definecolor{dkgreen}{rgb}{0,0.6,0}
\definecolor{gray}{rgb}{0.5,0.5,0.5}
\definecolor{mauve}{rgb}{0.58,0,0.82}

\lstset{frame=tb,
  language=Java,
  aboveskip=3mm,
  belowskip=3mm,
  showstringspaces=false,
  columns=flexible,
  basicstyle={\small\ttfamily},
  numbers=none,
  numberstyle=\tiny\color{gray},
  keywordstyle=\color{blue},
  commentstyle=\color{dkgreen},
  stringstyle=\color{mauve},
  breaklines=true,
  breakatwhitespace=true,
  tabsize=3
}

\title{Round Robin}								% Title
\author{Maharaj Faawwaz A Yusran\\ Ayu Nabilah Binti Rozani\\ Nur Hajidah Iffah Binti Abdul Rahim\\ Subhashini Kannan}								% Author
\date{\today}											% Date

\makeatletter
\let\thetitle\@title
\let\theauthor\@author
\let\thedate\@date
\makeatother

\pagestyle{fancy}
\fancyhf{}
%\rhead{\theauthor}
\lhead{\thetitle}
\cfoot{\thepage}


\begin{document}

%%%%%%%%%%%%%%%%%%%%%%%%%%%%%%%%%%%%%%%%%%%%%%%%%%%%%%%%%%%%%%%%%%%%%%%%%%%%%%%%

\begin{titlepage}
	\centering
    \vspace*{0.5 cm}
    \includegraphics[scale = 0.25]{logoumt.png}\\[1.0 cm]	% University Logo
    \textsc{\LARGE University Malaysia Terengganu}\\[1.0 cm]	% University Name
	\textsc{\Large CSF3305}\\[0.4 cm]				% Course Code
	\textsc{\large Operating System}\\[0.6 cm]				% Course Name
	\rule{\linewidth}{0.2 mm} \\[0.4 cm]
	{ \huge \bfseries \thetitle}\\
	\rule{\linewidth}{0.2 mm} \\[1.5 cm]
	
	\begin{minipage}{0.75\textwidth}
		\begin{flushleft} \large
			\emph{Author:}\\
			\theauthor
			\end{flushleft}
			\end{minipage}~
			\begin{minipage}{0.3\textwidth}
			\begin{flushleft} \large
			\emph{Matric Number:} \\
			       S52500\newline
			       S51356\newline
						 S50947\newline
						 S52014\newline % Your Student Number
		\end{flushleft}
	\end{minipage}\\[3 cm]
 
	\vfill
	
\end{titlepage}

%%%%%%%%%%%%%%%%%%%%%%%%%%%%%%%%%%%%%%%%%%%%%%%%%%%%%%%%%%%%%%%%%%%%%%%%%%%%%%%%

\tableofcontents
\pagebreak

%%%%%%%%%%%%%%%%%%%%%%%%%%%%%%%%%%%%%%%%%%%%%%%%%%%%%%%%%%%%%%%%%%%%%%%%%%%%%%%%

\section{Introduction} 
An operating system needs a \textit{program scheduler} to schedule processes to
run on the computer. There are three types of schedulers: long term,
mid-term and short term. Moreover, an operating system uses two types
of scheduling process executions which are preemptive and non-preemptive. In
preemptive scheduling policy, a low priority process has to be suspended during
its execution if a higher priority process is waiting in the same queue for its
execution. In non-preemptive scheduling policy, processes are executed in
first come first serve basis, which means the next process is executed only when
the current running process finishes its execution.

There are a variety of ways to schedule processes that should run on the
computer, called scheduling algorithms. These scheduling algorithms are: First
Come First Serve (FCFS), Priority-based Scheduling, Shortest Job First (SJF),
Longest Job First (LJF), Shortest Remaining Time First (SRTF), Highest Response
Ratio Next and last but not least, the chosen topic of discussion: Round Robin
(RR).

To measure the performance of the algorithms, there are two main variables are
taken into account: \textit{turnaround time} and \textit{waiting time}, that are
calculated based on each process's \textit{arrival}, {burst} and
\textit{completion} time.

\textbf{Arrival time} is time at which the process arrives in the ready queue.
Completion time is time at which process complete its execution.

\textbf{Completion time} is time at which process complete its execution. 

\textbf{Burst time} is time required by a process for CPU execution.

To calculate \textit{turnaround time} and \textit{waiting time}, the following
formulas are used:

\[ Turn Around Time = Completion Time - Arrival Time \]

\[ Waiting Time = Turn Around Time - Burst Time \]

 \newpage 
 \section{Definition, Advantages and Disadvantages}

The Round Robin scheduling is simple, easy to implement, and starvation-free as
all processes get fair share of CPU. It is particularly effective in a
general-purpose time-sharing system or transaction processing system. It is also
one of the most commonly used technique in CPU scheduling as a core.

The advantages of Round Robin scheduling is all the processes have the equal
priority because of fixed time quantum. Starvation will never occur because each
process in every Round Robin scheduling cycle will be schedule for a fixed time
slice or time quantum.

The disadvantages of it is more overhead of context switching. In the Round
Robin scheduling algorithm, as the time quantum decreases context switching
increases. The increases in time quantum value results in time starvation which
may put many processes on hold. If the time quantum decreases, it will affect
the CPU efficiency. So, time quantum should neither be large nor small. If time
quantum becomes infinity, Round Robin scheduling algorithm gradually become
\textit{First Come First Serve} (FCFS) scheduling algorithm.

\newpage
\section{Source Code}

To understand the Round Robin scheduling algorithm, we have made a small
command-line project written in \textit{Java}.

The Java application takes in a list of processes and their respective
\textit{burst time} and \textit{arrival time} that is keyed in by the user, and
computes the order that processes should be executed in, by using the RR
algorithm. Next, the program calculates the \textit{turnarund time} and
\textit{waiting time} for each process and displays them in the standard output.

Firstly, we create a \textbf{Process} class which contains all the attributes of
a process running on the computer. Using the object-oriented way, we can store
an array of process objects later on in the main program. \textbf{Process.java}
is as follows:

\begin{lstlisting}
public class Process {

	private final int burstTime;
	private int remainingBurstTime, arrivalTime, timeArrivedInQueue, finishedTime;

	public Process(int burstTime, int arrivalTime) {
			this.burstTime = burstTime;
			this.remainingBurstTime = burstTime;
			this.arrivalTime = arrivalTime;
			this.finishedTime = 0;
	}

	public int getBurstTime() {
			return burstTime;
	}

	public int getRmBurstTime() {
			return remainingBurstTime;
	}

	public void setRmBurstTime(int burstTime) {
			this.remainingBurstTime = burstTime;
	}

	public void decreaseBurstTime(int quantum) {
			this.remainingBurstTime -= quantum;
	}

	public int getArrivalTime() {
			return arrivalTime;
	}

	public void setArrivalTime(int arrivalTime) {
			this.arrivalTime = arrivalTime;
	}

	public void setTimeArrivedInQueue(int t) {
			timeArrivedInQueue = t;
	}

	public void setFinishedTime(int finishedTime) {
			// only set the completion time if it's not already been set
			if (this.finishedTime == 0) {
					this.finishedTime = finishedTime;
			}
	}

	public int getTurnaroundTime() {
			// Using the formula
			return (int) Math.abs(finishedTime - timeArrivedInQueue);
	}

	public int getWaitingTime() {
			// Using the formula
			return getTurnaroundTime() - burstTime;
	}
}
\end{lstlisting}

Now we see the algorithm in action at \textbf{RoundRobinJava.java}:

\begin{lstlisting}
import java.util.Scanner;

public class RoundRobinJava {

	public static void main(String[] args) {
		Scanner sc = new Scanner(System.in);

		// Get input from user
		System.out.print("Number of processes: ");
		int numberOfProcesses = sc.nextInt();
		System.out.print("Quantum: ");
		int quantum = sc.nextInt();

		// Creates an empty array of processes to be filled in later
		Process[] processes = new Process[numberOfProcesses];

		// Initialize total time taken for all processes to be executed
		int totalTime = 0;

		// Get arrival times and burst times from user, loop through each
		// process
		for (int i = 0; i < numberOfProcesses; i++) {
			System.out.print("Arrival time for P" + i + " (lowest 0): ");
			int arrivalTime = sc.nextInt();
			System.out.print("Burst time for P" + i + " (lowest 0) : ");
			int burstTime = sc.nextInt();

			// Sets the ith process based on user input
			processes[i] = new Process(burstTime, arrivalTime);

			totalTime += processes[i].getBurstTime();
		}

		sc.close();

		// Displays initial data
		System.out.println("Process Num\t| Arrival\t| Burst");
		for (int i = 0; i < numberOfProcesses; i++) {
			System.out.println(displayProcessDetails(i, processes[i].getArrivalTime(), processes[i].getRmBurstTime()));
		}
		System.out.println();

		// Store waiting and turnaround times
		int[] waitingTimes = new int[numberOfProcesses];
		int[] turnaroundTimes = new int[numberOfProcesses];

		// CPU begins executing processes
		int time = 0;
		while (time < totalTime) {
			// loop through each process, check arrival and burst
			for (int num = 0; num < processes.length; num++) {
				// Check to see if the current process has arrived
				if (processes[num].getArrivalTime() <= time) {
					// Check that it still has remaining burst time
					if (processes[num].getRmBurstTime() >= quantum) {
						// Check if the process is executed for the first time
						// (burst time isn't decreased yet)
						if (processes[num].getRmBurstTime() ==
								processes[num].getBurstTime()) {
							// If yes, store it
							processes[num].setTimeArrivedInQueue(time);
						}
						// Add process to timeline
						printProcess(num);

						// Decrease current process's burst time by quantum
						processes[num].decreaseBurstTime(quantum);

						if (processes[num].getRmBurstTime() == 0) {
							// No burst left, set completion time
							processes[num].setFinishedTime(time);
							// calculate waiting and turnaround time
							turnaroundTimes[num] = processes[num].getTurnaroundTime();
							waitingTimes[num] = processes[num].getWaitingTime();
						}
					}
					// 1 burst finished, move on
					time += quantum;
				}
			}
		}

		// Display output
		System.out.println("\n\nTotal time: " + totalTime + "s\n");

		int totalWaiting = 0, totalTurnaround = 0;
		// Display turnaround and waiting times
		System.out.println("Process Num\t| Arrival\t| Burst \t| Waiting time \t| Turnaround time");
		for (int i = 0; i < numberOfProcesses; i++) {
			System.out.println(displayProcessResults(i, processes[i].getArrivalTime(), processes[i].getBurstTime(),
						waitingTimes[i], turnaroundTimes[i]));

			// Also sum up the times, to calculate average
			totalWaiting += waitingTimes[i];
			totalTurnaround += turnaroundTimes[i];
		}
		// Calculate and show the averages
		double avgWaitingTime = totalWaiting / numberOfProcesses;
		double avgTurnaroundTime = totalTurnaround / numberOfProcesses;
		System.out.println();
		System.out.printf("Average waiting time: %.2fs\n", avgWaitingTime);
		System.out.printf("Average turnaround time: %.2fs\n", avgTurnaroundTime);
	}
	/**
		* Plot the process to the ordered timeline
		*/
	private static void printProcess(int processIndex) {
		System.out.print("P" + processIndex + " - ");
	}
	/**
		* Display a process's details in table format
		*
		* @return process details as a String
		*/
	private static String displayProcessDetails(int i, int arrival, int burst) {
		return "P" + i + "\t\t| " + arrival + "\t\t| " + burst;
	}
	/**
		* Display a process's waiting and turnaround time in a table format
		*
		* @param i          the process number
		* @param arrival    arrival time of the process
		* @param burst      burst time of the process
		* @param waiting    waiting time of the process
		* @param turnaround turnaround time
		* @return process details as a String
		*/
	private static String displayProcessResults(int i, int arrival, int burst,   int waiting, int turnaround) {
		return displayProcessDetails(i, arrival, burst) + "\t\t| " + waiting + "\t\t| " + turnaround;
	}
}
\end{lstlisting}

\newpage
\section{Output and Discussion}

For quantum = 1, here is an example of the output:



\newpage
\section{Conclusion}
Summarize your work.


\newpage
\bibliographystyle{plain}
\bibliography{biblist}

Tutorialspoint.com (2019). Tutorials Point: Operating System Scheduling
algorithms. Retrieved 12 May, 2019, from https://www.tutorialspoint.com

W3schools (2019). Scheduling Algorithms of Operating System. Retrieved 12 May,
2019, from https://www.w3schools.in

GeeksforGeeks (2019). Program for Round Robin scheduling | Set 1. Retrieved 12
May, 2019, from https://www.geeksforgeeks.org

GeeksforGeeks (2019). Round Robin Scheduling with different arrival times.
Retrieved 12 May, 2019, from https://www.geeksforgeeks.org

\end{document}
