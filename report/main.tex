\documentclass[12pt]{article}
\usepackage[english]{babel}
\usepackage{natbib}
\usepackage{url}
\usepackage[utf8x]{inputenc}
\usepackage{amsmath}
\usepackage{graphicx}
\graphicspath{{image/}}
\usepackage{parskip}
\usepackage{fancyhdr}
\usepackage{vmargin}
\usepackage{listings}
\usepackage{color}

\setmarginsrb{3 cm}{2.5 cm}{3 cm}{2.5 cm}{1 cm}{1.5 cm}{1 cm}{1.5 cm}

\usepackage{listings}
\usepackage{color}
\definecolor{dkgreen}{rgb}{0,0.6,0}
\definecolor{gray}{rgb}{0.5,0.5,0.5}
\definecolor{mauve}{rgb}{0.58,0,0.82}

\lstset{frame=tb,
  language=Java,
  aboveskip=3mm,
  belowskip=3mm,
  showstringspaces=false,
  columns=flexible,
  basicstyle={\small\ttfamily},
  numbers=none,
  numberstyle=\tiny\color{gray},
  keywordstyle=\color{blue},
  commentstyle=\color{dkgreen},
  stringstyle=\color{mauve},
  breaklines=true,
  breakatwhitespace=true,
  tabsize=3
}

\title{Round Robin}								% Title
\author{Maharaj Faawwaz A Yusran\\ Ayu Nabilah Binti Rozani\\ Nur Hajidah Iffah Binti Abdul Rahim\\ Subhashini Kannan}								% Author
\date{\today}											% Date

\makeatletter
\let\thetitle\@title
\let\theauthor\@author
\let\thedate\@date
\makeatother

\pagestyle{fancy}
\fancyhf{}
%\rhead{\theauthor}
\lhead{\thetitle}
\cfoot{\thepage}


\begin{document}

%%%%%%%%%%%%%%%%%%%%%%%%%%%%%%%%%%%%%%%%%%%%%%%%%%%%%%%%%%%%%%%%%%%%%%%%%%%%%%%%

\begin{titlepage}
	\centering
    \vspace*{0.5 cm}
    \includegraphics[scale = 0.25]{logoumt.png}\\[1.0 cm]	% University Logo
    \textsc{\LARGE University Malaysia Terengganu}\\[1.0 cm]	% University Name
	\textsc{\Large CSF3305}\\[0.4 cm]				% Course Code
	\textsc{\large Operating System}\\[0.6 cm]				% Course Name
	\rule{\linewidth}{0.2 mm} \\[0.4 cm]
	{ \huge \bfseries \thetitle}\\
	\rule{\linewidth}{0.2 mm} \\[1.5 cm]
	
	\begin{minipage}{0.75\textwidth}
		\begin{flushleft} \large
			\emph{Author:}\\
			\theauthor
			\end{flushleft}
			\end{minipage}~
			\begin{minipage}{0.3\textwidth}
			\begin{flushleft} \large
			\emph{Matric Number:} \\
			       S52500\newline
			       S51356\newline
						 S50947\newline
						 S52014\newline % Your Student Number
		\end{flushleft}
	\end{minipage}\\[3 cm]
 
	\vfill
	
\end{titlepage}

%%%%%%%%%%%%%%%%%%%%%%%%%%%%%%%%%%%%%%%%%%%%%%%%%%%%%%%%%%%%%%%%%%%%%%%%%%%%%%%%

\tableofcontents
\pagebreak

%%%%%%%%%%%%%%%%%%%%%%%%%%%%%%%%%%%%%%%%%%%%%%%%%%%%%%%%%%%%%%%%%%%%%%%%%%%%%%%%

\section{Introduction} 
An operating system needs a \textit{program scheduler} to schedule processes to
run on the computer. There are three types of schedulers: long term,
mid-term and short term. Moreover, an operating system uses two types
of scheduling process executions which are preemptive and non-preemptive. In
preemptive scheduling policy, a low priority process has to be suspended during
its execution if a higher priority process is waiting in the same queue for its
execution. In non-preemptive scheduling policy, processes are executed in
first come first serve basis, which means the next process is executed only when
the current running process finishes its execution.

There are a variety of ways to schedule processes that should run on the
computer, called scheduling algorithms. These scheduling algorithms are: First
Come First Serve (FCFS), Priority-based Scheduling, Shortest Job First (SJF),
Longest Job First (LJF), Shortest Remaining Time First (SRTF), Highest Response
Ratio Next and last but not least, the chosen topic of discussion: Round Robin
(RR).

To measure the performance of the algorithms, there are two main variables are
taken into account: \textit{turnaround time} and \textit{waiting time}, that are
calculated based on each process's \textit{arrival}, {burst} and
\textit{completion} time.

\textbf{Arrival time} is time at which the process arrives in the ready queue.
Completion time is time at which process complete its execution.

\textbf{Completion time} is time at which process complete its execution. 

\textbf{Burst time} is time required by a process for CPU execution.

To calculate \textit{turnaround time} and \textit{waiting time}, the following
formulas are used:

\[ Turn Around Time = Completion Time - Arrival Time \]

\[ Waiting Time = Turn Around Time - Burst Time \]

 \newpage 
 \section{Description, Advantages and Disadvantages}

The Round Robin scheduling is simple, easy to implement, and starvation-free as
all processes get fair share of CPU. It is particularly effective in a
general-purpose time-sharing system or transaction processing system. It is also
one of the most commonly used technique in CPU scheduling as a core.

The advantages of Round Robin scheduling is all the processes have the equal
priority because of fixed time quantum. Starvation will never occur because each
process in every Round Robin scheduling cycle will be schedule for a fixed time
slice or time quantum.

The disadvantages of it is more overhead of context switching. In the Round
Robin scheduling algorithm, as the time quantum decreases context switching
increases. The increases in time quantum value results in time starvation which
may put many processes on hold. If the time quantum decreases, it will affect
the CPU efficiency. So, time quantum should neither be large nor small. If time
quantum becomes infinity, Round Robin scheduling algorithm gradually become
\textit{First Come First Serve} (FCFS) scheduling algorithm.

\newpage
\section{Source Code}

\begin{lstlisting}
	// Hello.java
	import javax.swing.JApplet;
	import java.awt.Graphics;
	
	public class Hello extends JApplet {
			public void paintComponent(Graphics g) {
					g.drawString("Hello, world!", 65, 95);
			}    
	}
	\end{lstlisting}

\newpage
\section{Output and Discussion}
Present the outputs here.

\newpage
\section{Conclusion}
Summarize your work.


\newpage
\bibliographystyle{plain}
\bibliography{biblist}

\end{document}
